\documentclass[letterpaper,12pt]{article}
\usepackage[utf8]{inputenc}
\usepackage[T1]{fontenc}
\usepackage[top=1.5cm, bottom=1.5cm, left=1.5cm, right=1.5cm]{geometry}
\usepackage{longtable}
\usepackage{hyperref}
\usepackage{booktabs}
\title{Details about the cost assigment for Mexico's power plants for the implementation of SWITCH Model}
\usepackage{authblk}
\author[1,2]{Aldo Sayeg Pasos Trejo}
\affil[1]{\textit{Physics Departament. Facultad de Ciencias. Universidad Nacional Autonoma de Mexico}}
\affil[2]{\textit{Visiting Student Researcher for the Berkeley Energy and Climate Institute at University of California, Berkeley}}
\date{\today}
\begin{document}
\maketitle
\section{Introduction}
This documents discusses the costs estimation for selected elecrticity power plants of Mexico's SEN("Sistema eléctrico nacional", National electrical system). This data is contained in the tables located at the "Generation/data" folder of the SWITCH-Mexico repository\cite{repo} for the implementation of the SWITCH model for Mexico's electricity grid made at the University of California, Berkeley.
\section{Plant selection}
The source for the number of plants existing in the system is the PIIRCE ("Programa Indicativo para la Instalación y Retiro de Centrales Eléctricas ", Program for Power Plant Installation and Decommission). The PIIRCE, a part of the PRODESEN ("Programa de Desarrollo del Sistema Eléctrico Nacional", Developement Program for the National Electric System), publishes a list\cite{piirce} of existing and proposed (with more specific details about their status) power plants in Mexico's electricity grid. Several copies with small changes from the original cam be found in the GitHub repository.
\\
\\ A translated and refined version of this table can be found in the "Generation/data" folder, with the name "PowerPlants.csv". In this table, plants are asociated with it's generation technology, load area and other things. PIIRCE's contains power plants that do not exist yet, but are being considered to be built. The "being\_built" column of the "GenerationPlants.csv" table explains the status of the plant. The following is a list of the terms in this column and it's meaning:
\begin{itemize}
\item \textbf{operational: } An existing and functional power plant.
\item \textbf{firm\_project:} Non existing power plant. A project that has been aproved by CENACE's ("Centro Nacional de Control de Energía", National Center for Energetic Control) criterion.
\item \textbf{optimization:} Non existing power plant. A project that has \textbf{not} been aproved by CENACE's ("Centro Nacional de Control de Energía", National Center for Energetic Control) criterion.
\item \textbf{generic\_project:} Non existing power plant. A project that has been proposed but without reviewn CENACE's criterion. Basically, suggested projects.
\item \textbf{auction\_projects:} Non existing power plant. Projects that have planed by private indusrtries that have gained permission to build a power plant through an goverment-sponsored auction.
\item \textbf{rehabilitation\_modernization:} existing power plant that will be upgraded and updated in it's infrastructure.
\end{itemize}
We estimated the costs only for power plants with a "being\_built" value of operational, firm\_project, rehabilitation\_modernization and auction\_projects. We made this choice as this are the only power plants that are built or that have a high chance of being built, and also to include a projection of the influence of the private sector power plants in the electricity grid.
\section{Cost assignment sources}
The Original PIIRCE power plant table already has a cost assigned to fixed and variable O \& M and to overnigh cost. However, we find that this cost does not fit or accomodates itself with international standards and other things. That is the main reason for making this estimation.
\\
\\This estimation will asign the mentioned costs to the selected power plants based on it's capacity and generation technology. We used as source of the costs of each plant the data contained in the COPAR ("Costos y parámetros de Referencia para la formulación de Proyectos de Inversión en el Sector Eléctrico", Reference costs ) generation report, which can be found in the "Generation/data" folder\cite{copar}.
\\
\\This document contains data refering to the O \& M and overnight costs of power plants of a determined capacity and technology in Mexico. However, this document doesn't has information for plants that use biofuels as their technology. To get an estimate of the costs of these plants, we used the data of the SWITCH model implementation for the West Coast of the United States. The report\cite{us} of this implementation contains the needed costs for each plant using biofuels.
\\
\\The "TechCosts.csv" file of the "Generation/data" folder contains this information.
\section{Python Script and mathematical details}
Using as a base the "PowerPlants.csv" table, the "Cost assignment" python script, located in the "Generation folder", assigned a fixed and variable O\&M cost and a overnight cost for all the selected power plants. For this, the script assignated the price of the "TechCosts.csv" table for the plant that had the neares value of capacity for the technology of the plant. For example, if a given selected plant is based on "combined\_cycle\_natural\_gas" tech and has a capacity of 260 MW, the script will assign to its costs the one of the plant with the same tech and a capacity of 285.7 MW that is pointed out in the "TechCosts.csv" table. 
\\
\\After assigning all of this information to the selected plants, it exports a table to the "Generation/data" folder called "PowerPlantsWithCost.csv". The units of the cost columns are the following:
\begin{center}
\begin{tabular}{|c|c|}
\hline
Column & unit \\
\hline
fixed\_o\_m & USD/MWh \\
variable\_o\_m & USD/MW-year \\
overnight\_cost & USD/KW \\
\hline
\end{tabular}
\end{center}
\begin{thebibliography}{9}
\bibitem{repo} Switch-Mexico data repository. \url{https://github.com/sergiocastellanos/switch_mexico_data}
\bibitem{piirce}
PIIRCE's power plants database. \url{base.energia.gob.mx/prodesen/BasedeDatos_PIIRCE-2016-2030_Generacion.xlsx}
\bibitem{copar} COPAR Generation report \url{https://github.com/sergiocastellanos/switch_mexico_data/tree/master/Generation/data/COPARGeneration}
\bibitem{us}Timescales of power system balancing for decarbonization of the electricty sector. Supplemental Information \url{https://github.com/sergiocastellanos/switch_mexico_data/tree/master/Generation/data/SWITCHusinfo}
\end{thebibliography}
\end{document}